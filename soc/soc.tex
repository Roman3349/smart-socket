\documentclass[12pt,a4paper,oneside]{article}
\usepackage[utf8]{inputenc}
\usepackage[czech]{babel}
\usepackage[T1]{fontenc}
\usepackage{graphicx}
\author{Roman Ondráček}
\title{Dálkové ovládání domácích spotřebičů mobilním telefonem}
\begin{document}
% Řádkování 1,5
\renewcommand{\baselinestretch}{1.5}
% Vynechání číslování
\pagestyle{empty}
% Zvětšení oblasti tisku pro tuto stránku
\enlargethispage{60mm}
\begin{center}
\large \textbf{STŘEDOŠKOLSKÁ ODBORNÁ ČINNOST} \\

\vspace{32mm}
% Český název
\huge \textbf{Dálkové ovládání domácích spotřebičů mobilním telefonem} \\

\vspace{64mm}
\end{center}

\begin{tabbing}
% Nastavení zarážek
\hspace{4mm} \= \hspace{24mm}  \=   \kill
\> \large \textbf{Autor:}  \> \large{Roman Ondráček}                                    \\[4mm]
\> \large \textbf{Škola:}  \> \large{Gymnázium Boskovice, příspěvková organizace}       \\[4mm]
\> \large \textbf{Kraj:}   \> \large{Jihomoravský}                                      \\[4mm]
\> \large \textbf{Obor:}   \> \large{10. Elektrotechnika, elektronika a telekomunikace} \\[24mm]
\> \large \textbf{Boskovice 2017}
\end{tabbing}

\newpage

% Řádkování 1,5
\renewcommand{\baselinestretch}{1.5}
% Vynechání číslování
\pagestyle{empty}
% Zvětšení oblasti tisku pro tuto stránku
\enlargethispage{60mm}
\begin{center}
\large \textbf{STŘEDOŠKOLSKÁ ODBORNÁ ČINNOST} \\

\vspace{32mm}
% Český název
\huge \textbf{Dálkové ovládání domácích spotřebičů mobilním telefonem} \\
\vspace{16mm}
% Anglický název
\huge \textbf{The remote control of home appliances via a mobile phone} \\

\vspace{24mm}
\end{center}

\begin{tabbing}
% Nastavení zarážek
\hspace{4mm} \= \hspace{24mm}  \=   \kill
\> \large \textbf{Autor:}     \> \large{Roman Ondráček}                                    \\[4mm]
\> \large \textbf{Škola:}     \> \large{Gymnázium Boskovice, příspěvková organizace}       \\[4mm]
\> \large \textbf{Kraj:}      \> \large{Jihomoravský}                                      \\[4mm]
\> \large \textbf{Školitel:}  \> \large{prof. Ing. Václav Říčný, CSc.}                     \\[4mm]
\> \large \textbf{Obor:}      \> \large{10. Elektrotechnika, elektronika a telekomunikace} \\[16mm]
\> \large \textbf{Boskovice 2017}
\end{tabbing}

\normalsize

\newpage

~ \vspace{88mm}

\section*{Prohlášení}

Prohlašuji, že svou práci na téma Dálkové ovládání domácích spotřebičů mobilním telefonem jsem vypracoval samostatně pod vedením prof. Ing. Václava Říčného, CSc. a s použitím odborné literatury a dalších informačních zdrojů, které jsou všechny citovány v práci a uvedeny v seznamu literatury na konci práce. \\
Dále prohlašuji, že tištěná i elektronická verze práce SOČ jsou shodné a nemám závažný důvod proti zpřístupňování této práce v souladu se zákonem č.~121/2000 Sb., o právu autorském, o právech souvisejících s právem autorským a změně některých zákonů (autorský zákon) v platném změní. \\[8mm]

V Boskovicích dne \today \hspace{32mm} Podpis: 

\newpage

~ \vspace{88mm}

\section*{Poděkování}

Děkuji svému školiteli prof. Ing. Václavu Říčnému CSc. za obětavou pomoc a podnětné připomínky, které mi během práce poskytoval. \\
Tato práce byla provedena za finanční podpory Jihomoravského kraje.

\vspace{8mm}
% Loga
\begin{figure}[!htb]
\minipage{0.50\textwidth}
	\includegraphics[width = 64mm]{img/logo-jmk.pdf} \\[8mm]
	\includegraphics[width = 64mm]{img/logo-jcmm.jpg}
\endminipage
\minipage{0.50\textwidth}
	\includegraphics[width = 64mm]{img/logo-vut.pdf}
\endminipage
\end{figure}

\newpage

\section*{Anotace}

Cílem této práce je navrhnout a sestavit chytrou zásuvku, která se ovládá pomocí pomocí SMS.

\subsection*{Klíčová slova}

GSM; SMS;

\section*{Annotation}

The goal of this work is to design and build an smart power socket, which is controlled  by a  text message.

\subsection*{Keywords}

GSM; text message; 

\newpage

\tableofcontents


\newpage

\section*{Úvod}

\addcontentsline{toc}{section}{Úvod}

Text

\newpage

\section{Technické řešení}

\subsection{Blokové schéma}

\subsection{Obvodové schéma}

\subsection{Výkres plošného spoje a rozložení součástek}

\subsection{Rozpiska součástek}

\subsection{Software}

\newpage

\section*{Závěr}

\addcontentsline{toc}{section}{Závěr}

Text

\newpage

\begin{thebibliography}{99}

\bibitem{raspi}
\emph{Raspberry Pi} [online]. [cit. 2016-04-15]. Dostupné z: https://www.raspberrypi.org/

\end{thebibliography}

\newpage

\listoffigures

\listoftables

\end{document}